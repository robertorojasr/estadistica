% Generated by Sphinx.
\def\sphinxdocclass{report}
\documentclass[letterpaper,10pt,openany,oneside]{sphinxmanual}
\usepackage[utf8]{inputenc}
\DeclareUnicodeCharacter{00A0}{\nobreakspace}
\usepackage{cmap}
\usepackage[T1]{fontenc}
\usepackage[spanish]{babel}
\usepackage{times}
\usepackage[Sonny]{fncychap}
\usepackage{longtable}
\usepackage{sphinx}
\usepackage{multirow}


\addto\captionsspanish{\renewcommand{\figurename}{Figura }}
\addto\captionsspanish{\renewcommand{\tablename}{Tabla }}
\floatname{literal-block}{Lista }



\title{Resumen Estadística Documentation}
\date{29 de November de 2015}
\release{0}
\author{Roberto Rojas R.}
\newcommand{\sphinxlogo}{}
\renewcommand{\releasename}{Publicación}
\makeindex

\makeatletter
\def\PYG@reset{\let\PYG@it=\relax \let\PYG@bf=\relax%
    \let\PYG@ul=\relax \let\PYG@tc=\relax%
    \let\PYG@bc=\relax \let\PYG@ff=\relax}
\def\PYG@tok#1{\csname PYG@tok@#1\endcsname}
\def\PYG@toks#1+{\ifx\relax#1\empty\else%
    \PYG@tok{#1}\expandafter\PYG@toks\fi}
\def\PYG@do#1{\PYG@bc{\PYG@tc{\PYG@ul{%
    \PYG@it{\PYG@bf{\PYG@ff{#1}}}}}}}
\def\PYG#1#2{\PYG@reset\PYG@toks#1+\relax+\PYG@do{#2}}

\expandafter\def\csname PYG@tok@o\endcsname{\def\PYG@tc##1{\textcolor[rgb]{0.40,0.40,0.40}{##1}}}
\expandafter\def\csname PYG@tok@ne\endcsname{\def\PYG@tc##1{\textcolor[rgb]{0.00,0.44,0.13}{##1}}}
\expandafter\def\csname PYG@tok@cm\endcsname{\let\PYG@it=\textit\def\PYG@tc##1{\textcolor[rgb]{0.25,0.50,0.56}{##1}}}
\expandafter\def\csname PYG@tok@mf\endcsname{\def\PYG@tc##1{\textcolor[rgb]{0.13,0.50,0.31}{##1}}}
\expandafter\def\csname PYG@tok@gs\endcsname{\let\PYG@bf=\textbf}
\expandafter\def\csname PYG@tok@kp\endcsname{\def\PYG@tc##1{\textcolor[rgb]{0.00,0.44,0.13}{##1}}}
\expandafter\def\csname PYG@tok@k\endcsname{\let\PYG@bf=\textbf\def\PYG@tc##1{\textcolor[rgb]{0.00,0.44,0.13}{##1}}}
\expandafter\def\csname PYG@tok@vg\endcsname{\def\PYG@tc##1{\textcolor[rgb]{0.73,0.38,0.84}{##1}}}
\expandafter\def\csname PYG@tok@s\endcsname{\def\PYG@tc##1{\textcolor[rgb]{0.25,0.44,0.63}{##1}}}
\expandafter\def\csname PYG@tok@mo\endcsname{\def\PYG@tc##1{\textcolor[rgb]{0.13,0.50,0.31}{##1}}}
\expandafter\def\csname PYG@tok@sr\endcsname{\def\PYG@tc##1{\textcolor[rgb]{0.14,0.33,0.53}{##1}}}
\expandafter\def\csname PYG@tok@bp\endcsname{\def\PYG@tc##1{\textcolor[rgb]{0.00,0.44,0.13}{##1}}}
\expandafter\def\csname PYG@tok@nt\endcsname{\let\PYG@bf=\textbf\def\PYG@tc##1{\textcolor[rgb]{0.02,0.16,0.45}{##1}}}
\expandafter\def\csname PYG@tok@sd\endcsname{\let\PYG@it=\textit\def\PYG@tc##1{\textcolor[rgb]{0.25,0.44,0.63}{##1}}}
\expandafter\def\csname PYG@tok@kt\endcsname{\def\PYG@tc##1{\textcolor[rgb]{0.56,0.13,0.00}{##1}}}
\expandafter\def\csname PYG@tok@gu\endcsname{\let\PYG@bf=\textbf\def\PYG@tc##1{\textcolor[rgb]{0.50,0.00,0.50}{##1}}}
\expandafter\def\csname PYG@tok@nl\endcsname{\let\PYG@bf=\textbf\def\PYG@tc##1{\textcolor[rgb]{0.00,0.13,0.44}{##1}}}
\expandafter\def\csname PYG@tok@no\endcsname{\def\PYG@tc##1{\textcolor[rgb]{0.38,0.68,0.84}{##1}}}
\expandafter\def\csname PYG@tok@s2\endcsname{\def\PYG@tc##1{\textcolor[rgb]{0.25,0.44,0.63}{##1}}}
\expandafter\def\csname PYG@tok@kr\endcsname{\let\PYG@bf=\textbf\def\PYG@tc##1{\textcolor[rgb]{0.00,0.44,0.13}{##1}}}
\expandafter\def\csname PYG@tok@s1\endcsname{\def\PYG@tc##1{\textcolor[rgb]{0.25,0.44,0.63}{##1}}}
\expandafter\def\csname PYG@tok@ge\endcsname{\let\PYG@it=\textit}
\expandafter\def\csname PYG@tok@gi\endcsname{\def\PYG@tc##1{\textcolor[rgb]{0.00,0.63,0.00}{##1}}}
\expandafter\def\csname PYG@tok@gp\endcsname{\let\PYG@bf=\textbf\def\PYG@tc##1{\textcolor[rgb]{0.78,0.36,0.04}{##1}}}
\expandafter\def\csname PYG@tok@kd\endcsname{\let\PYG@bf=\textbf\def\PYG@tc##1{\textcolor[rgb]{0.00,0.44,0.13}{##1}}}
\expandafter\def\csname PYG@tok@nn\endcsname{\let\PYG@bf=\textbf\def\PYG@tc##1{\textcolor[rgb]{0.05,0.52,0.71}{##1}}}
\expandafter\def\csname PYG@tok@w\endcsname{\def\PYG@tc##1{\textcolor[rgb]{0.73,0.73,0.73}{##1}}}
\expandafter\def\csname PYG@tok@kc\endcsname{\let\PYG@bf=\textbf\def\PYG@tc##1{\textcolor[rgb]{0.00,0.44,0.13}{##1}}}
\expandafter\def\csname PYG@tok@nv\endcsname{\def\PYG@tc##1{\textcolor[rgb]{0.73,0.38,0.84}{##1}}}
\expandafter\def\csname PYG@tok@se\endcsname{\let\PYG@bf=\textbf\def\PYG@tc##1{\textcolor[rgb]{0.25,0.44,0.63}{##1}}}
\expandafter\def\csname PYG@tok@sb\endcsname{\def\PYG@tc##1{\textcolor[rgb]{0.25,0.44,0.63}{##1}}}
\expandafter\def\csname PYG@tok@ni\endcsname{\let\PYG@bf=\textbf\def\PYG@tc##1{\textcolor[rgb]{0.84,0.33,0.22}{##1}}}
\expandafter\def\csname PYG@tok@vc\endcsname{\def\PYG@tc##1{\textcolor[rgb]{0.73,0.38,0.84}{##1}}}
\expandafter\def\csname PYG@tok@mb\endcsname{\def\PYG@tc##1{\textcolor[rgb]{0.13,0.50,0.31}{##1}}}
\expandafter\def\csname PYG@tok@gd\endcsname{\def\PYG@tc##1{\textcolor[rgb]{0.63,0.00,0.00}{##1}}}
\expandafter\def\csname PYG@tok@mi\endcsname{\def\PYG@tc##1{\textcolor[rgb]{0.13,0.50,0.31}{##1}}}
\expandafter\def\csname PYG@tok@nb\endcsname{\def\PYG@tc##1{\textcolor[rgb]{0.00,0.44,0.13}{##1}}}
\expandafter\def\csname PYG@tok@nd\endcsname{\let\PYG@bf=\textbf\def\PYG@tc##1{\textcolor[rgb]{0.33,0.33,0.33}{##1}}}
\expandafter\def\csname PYG@tok@cp\endcsname{\def\PYG@tc##1{\textcolor[rgb]{0.00,0.44,0.13}{##1}}}
\expandafter\def\csname PYG@tok@nf\endcsname{\def\PYG@tc##1{\textcolor[rgb]{0.02,0.16,0.49}{##1}}}
\expandafter\def\csname PYG@tok@err\endcsname{\def\PYG@bc##1{\setlength{\fboxsep}{0pt}\fcolorbox[rgb]{1.00,0.00,0.00}{1,1,1}{\strut ##1}}}
\expandafter\def\csname PYG@tok@mh\endcsname{\def\PYG@tc##1{\textcolor[rgb]{0.13,0.50,0.31}{##1}}}
\expandafter\def\csname PYG@tok@c1\endcsname{\let\PYG@it=\textit\def\PYG@tc##1{\textcolor[rgb]{0.25,0.50,0.56}{##1}}}
\expandafter\def\csname PYG@tok@gt\endcsname{\def\PYG@tc##1{\textcolor[rgb]{0.00,0.27,0.87}{##1}}}
\expandafter\def\csname PYG@tok@na\endcsname{\def\PYG@tc##1{\textcolor[rgb]{0.25,0.44,0.63}{##1}}}
\expandafter\def\csname PYG@tok@c\endcsname{\let\PYG@it=\textit\def\PYG@tc##1{\textcolor[rgb]{0.25,0.50,0.56}{##1}}}
\expandafter\def\csname PYG@tok@cs\endcsname{\def\PYG@tc##1{\textcolor[rgb]{0.25,0.50,0.56}{##1}}\def\PYG@bc##1{\setlength{\fboxsep}{0pt}\colorbox[rgb]{1.00,0.94,0.94}{\strut ##1}}}
\expandafter\def\csname PYG@tok@ss\endcsname{\def\PYG@tc##1{\textcolor[rgb]{0.32,0.47,0.09}{##1}}}
\expandafter\def\csname PYG@tok@kn\endcsname{\let\PYG@bf=\textbf\def\PYG@tc##1{\textcolor[rgb]{0.00,0.44,0.13}{##1}}}
\expandafter\def\csname PYG@tok@gr\endcsname{\def\PYG@tc##1{\textcolor[rgb]{1.00,0.00,0.00}{##1}}}
\expandafter\def\csname PYG@tok@go\endcsname{\def\PYG@tc##1{\textcolor[rgb]{0.20,0.20,0.20}{##1}}}
\expandafter\def\csname PYG@tok@sc\endcsname{\def\PYG@tc##1{\textcolor[rgb]{0.25,0.44,0.63}{##1}}}
\expandafter\def\csname PYG@tok@ow\endcsname{\let\PYG@bf=\textbf\def\PYG@tc##1{\textcolor[rgb]{0.00,0.44,0.13}{##1}}}
\expandafter\def\csname PYG@tok@si\endcsname{\let\PYG@it=\textit\def\PYG@tc##1{\textcolor[rgb]{0.44,0.63,0.82}{##1}}}
\expandafter\def\csname PYG@tok@m\endcsname{\def\PYG@tc##1{\textcolor[rgb]{0.13,0.50,0.31}{##1}}}
\expandafter\def\csname PYG@tok@vi\endcsname{\def\PYG@tc##1{\textcolor[rgb]{0.73,0.38,0.84}{##1}}}
\expandafter\def\csname PYG@tok@il\endcsname{\def\PYG@tc##1{\textcolor[rgb]{0.13,0.50,0.31}{##1}}}
\expandafter\def\csname PYG@tok@gh\endcsname{\let\PYG@bf=\textbf\def\PYG@tc##1{\textcolor[rgb]{0.00,0.00,0.50}{##1}}}
\expandafter\def\csname PYG@tok@nc\endcsname{\let\PYG@bf=\textbf\def\PYG@tc##1{\textcolor[rgb]{0.05,0.52,0.71}{##1}}}
\expandafter\def\csname PYG@tok@sx\endcsname{\def\PYG@tc##1{\textcolor[rgb]{0.78,0.36,0.04}{##1}}}
\expandafter\def\csname PYG@tok@sh\endcsname{\def\PYG@tc##1{\textcolor[rgb]{0.25,0.44,0.63}{##1}}}

\def\PYGZbs{\char`\\}
\def\PYGZus{\char`\_}
\def\PYGZob{\char`\{}
\def\PYGZcb{\char`\}}
\def\PYGZca{\char`\^}
\def\PYGZam{\char`\&}
\def\PYGZlt{\char`\<}
\def\PYGZgt{\char`\>}
\def\PYGZsh{\char`\#}
\def\PYGZpc{\char`\%}
\def\PYGZdl{\char`\$}
\def\PYGZhy{\char`\-}
\def\PYGZsq{\char`\'}
\def\PYGZdq{\char`\"}
\def\PYGZti{\char`\~}
% for compatibility with earlier versions
\def\PYGZat{@}
\def\PYGZlb{[}
\def\PYGZrb{]}
\makeatother

\renewcommand\PYGZsq{\textquotesingle}

\begin{document}
\shorthandoff{"}
\maketitle
\tableofcontents
\phantomsection\label{index::doc}



\chapter{Introducción}
\label{Introduccion::doc}\label{Introduccion:introduccion}
La idea de este documento es resumir los apuntes y aglomerar información que puede estar
dispersa, abarca más o menos lo encontrado en los apuntes combinando información de
diapositivas y documentos.

\begin{notice}{warning}{Advertencia:}
El presente documento no pretende ser un texto de referencia ni ser infalible, es posible
que se me pase algún errorcillo, por eso se recomienda usar de guia para complementar los
apuntes y bibliografia del ramo, además este ramo lo hice hace como 10 años, la memoria es
frágil y mucho ha pasado por esta azotea.
\end{notice}

La estadística se compone de:
\begin{quote}
\begin{description}
\item[{Estadística \textbf{Descriptiva}}] \leavevmode
Recolectar, ordenar, resumir y representar. Solo muestra lo que hay, solo que más
bonito, legible y conciso. En esencia tomar una tabla detallada y monstruosa de Excel
y convertirla en unas lindas tablas resumidas y gráficos para el artículo, paper o
presentación de PowerPoint.

\item[{Estadística \textbf{Inferencial}}] \leavevmode
El inferir algo es proyectar lo general basandose en algo local, esta rama de la
estadística se encarga de hacerlo bien, no basta con decir que porque soy el más alto
de mi familia soy el más alto de Chile. Es labor de la estadística inferencial, por
ejemplo, saber si el promedio obtenido con estadística descriptiva es representativo o
no, si una muestra es más dispersa que otra, etc.
\begin{description}
\item[{\textbf{Paramétrica}}] \leavevmode
Los datos pueden distribuirse regularmente (generalmente lo estan), estas
distribuciones pueden definirse matemáticamente y como son regulares basta con
conocer a que distribución se asemejan y con unos pocos (en términos relativos)
parámetros se puede describir el conjunto de datos e incluso proyectar y predecir
futuras incidencias.

\item[{\textbf{NO paramétrica}}] \leavevmode
Cuando no conocemos a que distribución matematica se ajustan los datos, no se
ajusta bien a ninguna (raro) o ni se parece a alguna distribución (muuuy raro) se
utiliza la estadística NO parametrica (no conocemos los parametros).
Estas casi nunca se hacen a mano, asi que pese a que suelen ser más complejas (no
siempre y no todas) suelen hacerse por un computador. Solo es necesario saber cuando
aplicar cada una y como interpretar los resultados. El programa que se suele usar
para estos menesteres es el SPSS y su versión gratuita y OpenSource PSPP.

\end{description}

\end{description}
\end{quote}

Esto está en los apuntes, pero no me pareció muy claro, estas clasificaciones se inducen
pero no se explicitan (puede que lo haya hecho en clases)


\chapter{Introducción a la estadística}
\label{Estadistica::doc}\label{Estadistica:introduccion-a-la-estadistica}

\section{Qué es la estadística}
\label{Estadistica:que-es-la-estadistica}

\section{Método científico}
\label{Estadistica:metodo-cientifico}\begin{description}
\item[{\textbf{1.Detección y enunciado}}] \leavevmode
Definición clara y específica, que quede claro de que se está hablando y no quede logar
para ambigüedades.

\item[{\textbf{2. Formulación de hipótesis}}] \leavevmode
Una posible explicación, si tiene algún fundamento es menos probable perder el tiempo
comprobando tonteras.

\item[{\textbf{3. Consecuencia verificable}}] \leavevmode
Definir el como se va a comprobar la hipótesis, definiendo variables comprobables y que
eniquívocamente confirmen o descarten la hipótesis. si estas no están bien definidas voy
a hacer mucho trabajo en los pasos siguientes sin haber demostrado o desmentido nada.

\item[{\textbf{4. Verificación}}] \leavevmode
Recolectar datos, procesarlos y analizarlos para, basado en el punto anterior, confirmar
o refutar la hipótesis.

\item[{\textbf{5. Conclusión}}] \leavevmode
Presentar resultados, responder si se cumple o no lo definido en el primer punto.

\end{description}


\section{Método estadístico}
\label{Estadistica:metodo-estadistico}\begin{description}
\item[{\textbf{A.Planificación}}] \leavevmode\begin{description}
\item[{\textbf{1. Definición de objetivos}}] \leavevmode
Definición clara del objetivo, detalle del qué, cómo, dónde, cuándo y por qué, sin
lugar a interpretaciones ni mal entendidos.

\item[{\textbf{2. Universo de estudio}}] \leavevmode
De donde se va a sacar la información.

\item[{\textbf{3. Diseño de la muestra}}] \leavevmode
Como seguramente es imposible, al menos en la práctica, obtener TODA la información
existente, debemos tomar parte de ella, cuidando que represente lo mejor posible el
universo definido en el punto anterior.

\item[{\textbf{4. Definición de las unidades}}] \leavevmode
Definir caracteristicas del qué vamos a observar, unidades de medida, escalas de
clasificación.

\item[{\textbf{5. Preparación del plan}}] \leavevmode
Establecer el plan de tabulación y análisis, el cómo se presentará la información y
como se hará el análisis.

\end{description}

\item[{\textbf{B.Ejecución}}] \leavevmode\begin{description}
\item[{\textbf{1. Recolección}}] \leavevmode
La parte \emph{entretenida}, pararse en la calle a encuestar, pararse en un recinto a
contar gente o cronometrar cosas una y otra y otra vez.

\item[{\textbf{2. Elaboración}}] \leavevmode
Tomar los benditos datos del punto anterior, los datos que tomaron horas y horas
capturar y convertirlo en gráficos y tablas que hagan a los demás fácil el visualizar
y entender la información. Aquí es donde pasan de datos a información.

\item[{\textbf{3. Análisis}}] \leavevmode
Teniendo la información del punto anterior ahora se procesa y analiza para, con ellos,
confirmar o descartar las aseveraciones del punto 1.

\end{description}

\end{description}


\section{Elementos básicos}
\label{Estadistica:elementos-basicos}\begin{description}
\item[{Unidad de análisis**}] \leavevmode
Teniendo listo el punto 1 de la planificación se define qué es lo que sera observado.

\item[{\textbf{Atributos}}] \leavevmode
Definir las características relevantes de las unidades antes establecidas, si vamos a
medir inteligencia en personas de cierto lugar, salvo que sea una hipótesis muy \emph{exótica},
el tamaño de las manos seguramente no es relevante, aquí se establecen cuales
características si lo son.

\item[{\textbf{Variables}}] \leavevmode
El qué y como voy a expresar los mencionados atributos, que posibles valores van a
tener, las unidades de medida, etc.

\end{description}
\phantomsection\label{Estadistica:escalas-de-medida}\begin{description}
\item[{\textbf{Escalas de medida}}] \leavevmode
En esencia el tipo de variable, entre las que se distinguen 3 tipos:
\begin{description}
\item[{\textbf{Nominal}}] \leavevmode
Es solo una etiqueta no relacionable con otros valores solo sirve para diferenciar
estados de la variable.

\item[{\textbf{Ordinal}}] \leavevmode
A diferencia de las Nominales, las variables Ordinales son ordenables, conllevan una
relación con respecto a otras variables cada valor lo hace mas o menos que otras.

\item[{\textbf{Intervalar}}] \leavevmode
Además de ser ordenables pueden operarse matemáticamente, tienen una unidad de medida
y todos los elementos la comparten, estas pueden ser:
\begin{description}
\item[{\textbf{Discretas}}] \leavevmode
Números enteros, cantidad de huevos, cantidad de personas, etc. Seguramente podemos
considerar fracciones de personas pero seria poco ético andar partiendo personas.

\item[{\textbf{Continuas}}] \leavevmode
Consideran todos los posibles valores dentro de un intervalo, formalmente números
reales y complejos pero salvo que midamos el spin de partículas subatómicas o
las funciones de probabilidad del orbital de un electrón nos basta con saber que son
números decimales, entendiendo que entre el 0 y el 1 hay infinitos valores (los
intervalos siempre pueden partirse en 2).

\end{description}

\end{description}

\item[{\textbf{Población}}] \leavevmode
El conjunto de todos los elementos que nos interesan, el conjunto universo.

\item[{\textbf{Muestra}}] \leavevmode
Por razones prácticas no siempre puede usarse la información de cada elemento de la
población, por ejemplo para hacer un control de calidad de fósforos o cuando queremos
saber cuanto miden los Chilenos; en el primer caso la única forma de saber si un fósforo
esta bueno es póstumo, lo que hace ridículo probarlos todos y en el segundo, seria muy
caro y dificil medir a cada chileno. Lo importante es tratar de que la mustra represente
a la población lo mejor posible. Si pretendo analizar la altura de los chilenos entre
los jinetes en el hipódromo, seguramente no voy a tener una muestra representantiva.

\end{description}


\chapter{Estadística Descriptiva}
\label{Estadistica_Descriptiva:estadistica-descriptiva}\label{Estadistica_Descriptiva::doc}
Es la que consolida los datos, ordena, resume y representa. O sea describe entrega ciertas
características de la muestra sin decir mucho de que tan buen trabajo está haciendo con
respecto a la población, no hace predicciones ni proyecciones, solo da información sobre
los datos de la muestra.


\section{Conceptos básicos}
\label{Estadistica_Descriptiva:conceptos-basicos}

\subsection{Graficar y tabular}
\label{Estadistica_Descriptiva:graficar-y-tabular}
Graficar y tabular como todo tiene su ciencia y su arte, puede ser la diferencia entre
auditores interesados que aprenden y auditores que no entienden nada y se duermen. Además
si no se hace con cuidado puede distorsionar la información percibida. Lo dicen 8 de cada
10 dentistas.

Hay muchas formas de tabular y graficar datos, cada una tiene sus usos, gracias y
desgracias, lo primero es identificar que tipo de variable ({\hyperref[Estadistica:escalas-de-medida]{\emph{escalas de medida}}}) vamos a representar:


\subsubsection{Nominales (no ordenables)}
\label{Estadistica_Descriptiva:nominales-no-ordenables}

\paragraph{Tablas}
\label{Estadistica_Descriptiva:tablas}
Basta con cuantas veces se repite cada valor y eso seria, también se puede agregar el \% de cada uno.


\paragraph{Gráficos}
\label{Estadistica_Descriptiva:graficos}
Son preferibles los circulares, se pueden usar tambien de barras pero las variables al
no tener orden no se gana nada con ponelas en ejes y se pierde la visualización de
la proporción con el total. Con el circular el orden no importa (es un circulo) y no
solo se representan las relaciones entre las variables, también se representan sus
relaciones con respecto al total de yapa.


\subsubsection{Ordinales (ordenables)}
\label{Estadistica_Descriptiva:ordinales-ordenables}

\paragraph{Tablas}
\label{Estadistica_Descriptiva:id1}
Aquí como se pueden y deben ordenar hace sentigo agregar la columna de acumulado, en esta
se van sumando las frecuencias de las variables menores (recordar que siempre se ordena de
menor a mayor), así se tiene la gracia de poder ver rápidamente por ejemplo todos los que
miden menos de 1,75m sin necesidad de tomar una calculadora y sumar todas las frecuencias.


\paragraph{Gráficos}
\label{Estadistica_Descriptiva:id2}
Aquí se pueden usar de barra o circulares, pero suelen ser mejores los de barra por que:
\begin{itemize}
\item {} 
En el circular se pierde el orden (que con las nominales no teniamos pero ahora si),
podriamos acordar que partimos arriba y seguimos en sentido horario pero es trabajo
adicional innecesario y nos carga ese tipo de trabajo.

\item {} 
Si tenemos muchos posibles valores para las variables o simplificamos los datos y
agrupamos por rango, perdiendo información o nos va a quedar un gráfico circular ilegible,
diferenciar el ancho de una tajada de un 1\% y una de un 3\% es bastante dificil y hasta los
colores y su posición relativa a los vecinos alterna la percepción del área de dicha
tajada. Con un grafico de barras basta con ensanchar el gráfico y adelgazar las barras
(acá es el alto el que nos da la información no el ancho).

\end{itemize}


\subsubsection{Intervalares (ordenable Y operables)}
\label{Estadistica_Descriptiva:intervalares-ordenable-y-operables}

\paragraph{Tablas}
\label{Estadistica_Descriptiva:id3}
Aquí se pueden usar tablas parecidas a las Ordinales pero mejor usamos lo último de la
tecnología en tablas con el diagrama de tallo y hoja inventado en 1977 por Turkey. Con
esta maravilla de la tecnología se separan los valores por orden y se enlistan las
unidades o sea para:

\{ 12, 13, 13, 14, 15, 23, 23, 23, 45, 46, 46, 47 \}

queda

1 \textbar{} 2 3 3 4 5
2 \textbar{} 3 3 3
4 \textbar{} 5 6 6 7

y con:

\{ 123, 123, 124, 134, 135, 135, 136, 137, 141, 141, 142 \}

queda
\begin{description}
\item[{1 \textbar{} 2 \textbar{} 3 3 4}] \leavevmode
\begin{DUlineblock}{0em}
\item[] 3 \textbar{} 4 5 5 6 7
\item[] 4 \textbar{} 1 1 2
\end{DUlineblock}

\end{description}

Notese que para hacer la vida más placentera ordenamos los datos antes de hacer cualquier
cosa.

La gracia de este tabulado es que no se pierde información, es fácil de interpretar y si
torcemos nuestras cabezas 90° en el sentido del reloj y voilá tenemos un grafico de
frecuencia agrupado por decenas.


\paragraph{Gráficos}
\label{Estadistica_Descriptiva:id4}
Se pueden mostrar:
\begin{description}
\item[{\textbf{Frecuencias}: cada barra muestra la frecuencia de dado valor en el valor de altura de 1.70m}] \leavevmode
va la frecuencia de las personas que miden 1.70.

\item[{\textbf{Frecuencias acumuladas}: cada barra indica su frecuencia y las frecuencias de los valores}] \leavevmode
más chicos por eso acumulada asi en la altura de 1.70m van todas
las personas que miden menos de 1.70m

\end{description}

Y para eso se usan los gráficos:

\textbf{Grafico de barras}: se usan para variables discretas y se dibujan con sus barras separadas
entre si.

\textbf{Histogramas}: se usan para variables continuas y se dibujan con las barras pegadas a sus
vecinos.

\textbf{Grafico de línea}: en vez de hacer rectángulos se tira una línea en el valor dado, cuando
es de frecuencias acumuladas se llama OJIVA, ni idea de porque se llama así. No se suelen
usar si son de frecuencias, se usa el de los anteriores que corresponda.

\begin{notice}{note}{Nota:}
Si al hacer la tabla se agrupan los datos en distintos rangos ej:

\begin{tabulary}{\linewidth}{|L|L|L|L|}
\hline

edad
 & 
freq
 & 
largo
 & 
freq*
\\
\hline
0 - 3
 & 
30
 & 
3
 & 
10
\\
\hline
3 - 5
 & 
50
 & 
2
 & 
25
\\
\hline
5 -10
 & 
40
 & 
5
 & 
8
\\
\hline
10 -30
 & 
60
 & 
20
 & 
3
\\
\hline\end{tabulary}


Vemos que en el primer rango 3 - 0 = 3, y el tercero 5 - 10 = 5, lo que se refleja en
la tercera columna, para ajustar las frecuencias y hacer que los gráficos que salgan de
aquí se vean representativos se divide la frecuencia por el largo y ESO se grafica.
\end{notice}


\subsection{Estadígrafos y estadisticos}
\label{Estadistica_Descriptiva:estadigrafos-y-estadisticos}

\section{Probabilidades}
\label{Estadistica_Descriptiva:probabilidades}

\section{Variables aleatorias DISCRETAS}
\label{Estadistica_Descriptiva:variables-aleatorias-discretas}

\section{Variables aleatorias CONTINUAS}
\label{Estadistica_Descriptiva:variables-aleatorias-continuas}

\chapter{Estadística inferencial}
\label{Estadistica_Inferencial::doc}\label{Estadistica_Inferencial:estadistica-inferencial}

\section{Estadística paramétrica}
\label{Estadistica_Inferencial:estadistica-parametrica}

\section{Estadística NO paramétrica}
\label{Estadistica_Inferencial:estadistica-no-parametrica}\begin{description}
\item[{\index{distribución|textbf}distribución, \index{función|textbf}función, \index{integral|textbf}integral, \index{sumatoria|textbf}sumatoria, \index{exponencial (e)|textbf}exponencial (e), \index{factorial (!)|textbf}factorial (!)}] \leavevmode\phantomsection\label{Glosario:term-distribucion}
\end{description}



\renewcommand{\indexname}{Índice}
\printindex
\end{document}
