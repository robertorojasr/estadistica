\documentclass[10pt,letterpaper,oneside]{report}
\usepackage{color}
\usepackage{mathtools}
\usepackage{hyperref}
\usepackage[utf8]{inputenc}

\begin{document}
\title{Resumen estadística}
\author{Roberto Rojas R.}
\date{\today}
\maketitle

\chapter{Introducción a la estadística}
  \subsection{Que es la estadística}

  \subsection{Método científico}
    \begin{description}
      \item[1. Detección y enunciado] \hfill \\
      \item[2. Formulación de hipótesis] \hfill \\
      \item[3. Consecuencia verificable] \hfill \\
      \item[4. Verificación] \hfill \\
      \item[5. Conclusión] \hfill \\
    \end{description}

  \subsection{Método estadístico}
    \begin{description}
      \item[A.Planificación] \hfill \\
        \begin{description}
          \item[1. Definición de objetivos] \hfill \\
          \item[2. Universo de estudio] \hfill \\
          \item[3. Diseño de la muestra] \hfill \\
          \item[4. Definición de las unidades] \hfill \\
          \item[5. Preparación del plan] \hfill \\
        \end{description}
      \item[B.Ejecución] \hfill \\
        \begin{description}
          \item[1. Recolección] \hfill \\
          \item[2. Elaboración] \hfill \\
          \item[3. Análisis] \hfill \\
        \end{description}
    \end{description}

  \subsection{Elementos básicos}
    \begin{description}
      \item[Unidad de analisis]
      \item[Variables]
      \item[Escalas de medida]
      \item[Escalas de población]
      \item[Escalas de muestra]
    \end{description}


\chapter{Estadística descriptiva}
  \section{Nociones de probabilidades}
  \section{Variables aleatórias}


\chapter{Estadística inferencial}
  \section{Estadística paramétrica}
  \section{Estadística no paramétrica}


\end{document}
